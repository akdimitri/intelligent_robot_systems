\section{Υλοποίηση Εργασίας.}

%--- Challenge 1 ---
\subsection{\textlatin{Challenge 1: Laser-based obstacle avoidance.}}

Για το ερώτημα 1 ζητείται να συμπληρωθεί κώδικας έτσι ώστε το σύστημα να υπολογίζει τη γραμμική (\textlatin{linear}) και τη γωνιακή (\textlatin{rational}) ταχύτητα (\textlatin{velocity}) χρησιμοποιώντας τις \textlatin{LIDAR} τιμές. Στόχος είναι το ρομπότ να περιπλανάται (\textlatin{wander}) χωρίς να συγκρούεται (\textlatin{collide}) σε εμπόδια.         

Από το μάθημα είναι γνωστό ότι: 

\begin{equation}
    u_{obs} = - \sum_{i=1}^{LaserRays}\frac{cos(θ_i)}{s_i^2}
\end{equation}

και

\begin{equation}
    ω_{obs} = - \sum_{i=1}^{LaserRays}\frac{sin(θ_i)}{s_i^2}
\end{equation}

όπου $θ_i$ η γωνία του \textlatin{Laser} και $s_i$ η απόσταση. \clearpage

Επομένως στη μέθοδο \textlatin{produceSpeedsLaser} έγιναν οι παρακάτω προσθήκες διασφαλίζοντας ότι η μέγιστη γραμμική ταχύτητα θα είναι $0.3 m/s$ και η μέγιστη γωνιακή $0.3 rad/s$:

\textlatin{
    \lstinputlisting[language=Python,firstline=3,lastline=21, basicstyle=\tiny]{./code.txt}
}

Στη συνέχεια, στη μέθοδο \textlatin{produceSpeeds} έγιναν οι παρακάτω προσθήκες:

\textlatin{
    \lstinputlisting[language=Python,firstline=23,lastline=44, basicstyle=\tiny]{./code.txt}
}

%--- Challenge 2 ---
\subsection{\textlatin{Challenge 2: Path visualization.}}

Ο σκοπός του ερωτήματος αυτού ήταν να γίνει εμφανές το \textlatin{path} στο \textlatin{RViz tool}. Συνεπώς, έγιναν οι εξής προσθήκες \textlatin{selectTarget}:

\textlatin{
    \lstinputlisting[language=Python,firstline=48,lastline=49, basicstyle=\tiny]{./code.txt}
}

Ουσιαστικά, πραγματοποιείται πολλαπλασιασμός του σημείου του \textlatin{path} με το \textlatin{resolution} και στη συνέχεια προστίθεται το \textlatin{origin}.

%--- Challenge 3 ---
\subsection{\textlatin{Challenge 3: Path following.}}

Ο σκοπός του ερωτήματος αυτού είναι να παραχθούν οι κατάλληλες ταχύτητες ώστε το ρομπότ να ακολουθήσει το \textlatin{path}. Από το μάθημα γνωρίζω ότι αν ξέρω την κατεύθυνση του ρομπότ και τη γωνία του αντικειμένου μπορώ να βρω τη σχετική μεταξύ τους γωνία και κατ' επέκταση τη ζητούμενη γωνιακή ταχύτητα ή:

\begin{equation}
    ω = \begin{cases}
        \frac{Δθ + 2π}{π}, \text{ αν } Δθ < -π \\
        \frac{Δθ}{π}, \text{ αν } -π \leq Δθ \leq π \\
        \frac{Δθ - 2π}{π}, \text{ αν } Δθ > π \\
    \end{cases}
\end{equation}

Τώρα, η τιμή της γραμμικής ταχύτητας υπολογίζεται ως εξής:

\begin{equation}
    linear_u = u_{max}(1- |ω|)^n \Leftrightarrow u = 0.3(1- |ω|)^n,
\end{equation}
όπου μετά από προσομοιώσεις επιλέχθηκε $n = 6$.

Ακόμη, επειδή η γωνιακή ταχύτητα δεν ήταν αρκετά μεγάλη και δεν έστριβε γρήγορα το ρομπότ, επιλέχθηκε να τροποποιείται ως εξής: $angular_u = 0.3*sgn(ω)ω^{6}$

Από τα παραπάνω γίνεται εμφανές ότι οι ταχύτητες ήταν ορισμένες στο διάστημα $[-0.3, 0.3]$.

Συμπληρώθηκε, λοιπόν, η μέθοδος \textlatin{velocitiesToNextSubtarget} με:

\textlatin{
    \lstinputlisting[language=Python,firstline=54,lastline=90, basicstyle=\tiny]{./code.txt}
}


%--- Challenge 4 ---
\subsection{\textlatin{Challenge 4: Path following \& obstacle avoidance.}}

Στο ερώτημα αυτό σκοπός είναι να ακολουθεί το ρομπότ μία διαδρομή αποφεύγοντας τα εμπόδια. Οι εξισώσεις που επιλέχθηκαν είναι:


\[ u = u_{path} + c_u*u_{obs} \]
\[ ω = ω_{path} + c_ω*ω_{obs} \]


προσδιορίζουμε, λοιπόν, τις τελικές ταχύτητες ως εξής:


\[ u = l_{goal} + l_{laser} * c1 \]
\[ ω = a_{goal} + a_{laser} * c2 \]

\[ u = min(0.3, max(-0.3, u)) \]
\[ ω = min(0.3, max(-0.3, ω)) \]

όπου $c_1 = 10^{-5}$ και $c2 = 10^{-5}$. Οι παράμετροι αυτοί προέκυψαν ύστερα από διάφορα πειράματα.

Οι προσθήκες που έγιναν στη μέθοδο \textlatin{produceSpeeds} είναι οι εξής:

\textlatin{
    \lstinputlisting[language=Python,firstline=94,lastline=109, basicstyle=\tiny]{./code.txt}
}


%--- Challenge 5 ---
\subsection{\textlatin{Challenge 5: Smarter subgoal checking.}}
Για το ερώτημα αυτό επιλέχθηκε να ελέγχεται η απόσταση από όλα τα επόμενα σημεία με μία πιο χαλαρή συνθήκη. Αν το ρομπότ βρίσκεται σχετικά σε μία κοντινή απόσταση από έναν στόχο, θεωρεί ότι τον έχει επιτύχει και συνεχίζει. Αν ακόμη παρατηρήσει ότι βρίσκεται κοντύτερα από κάποιο επόμενο στόχο (πάντα για μικρή απόσταση) προχωράει στον επόμενο στόχο. Ο \underline{μόνος} στόχος για τον οποίο η συνθήκη απόσταση είναι \underline{αυστηρή} είναι ο τελευταίος στόχος.

Έτσι η μέθοδος \textlatin{checkTarget} τροποποιήθηκε ως εξής:

\textlatin{
    \lstinputlisting[language=Python,firstline=113,lastline=135, basicstyle=\tiny]{./code.txt}
}


%--- Challenge 6 ---
\subsection{\textlatin{Challenge 6: Smart target selection.}}
Το ερώτημα αυτό αφορά την έξυπνη επιλογή ενός στόχου. Η μεθοδολογία που ακολουθήθηκε είναι αυτή που περιγράφεται στην παρουσίαση 9. Αρχικά, υπολογίζεται το \textlatin{path} για κάθε \textlatin{node}. Στη συνέχεια, προκειμένου να γίνει η "έξυπνη" επιλογή πρέπει να εισάγουμε κάποια μετρική-κόστος, έτσι υπολογίζεται το κόστος της απόστασης:

\[ w_{dist} = \sum_{i=1}^{PathSize-1} D_{i,i+1} \]

Επιπλέον, υπολογίζεται το τυπολογικό κόστος με τη χρήση της παρακάτω εξίσωσης:

\[ w_{topo} = brush(node) \]

όπου $brush$ η τιμή του \textlatin{brushfire} στο σημείο του  στόχου.

Ένα ακόμη κόστος που πρέπει να υπολογιστεί είναι το κόστος της περιστροφής και αυτό γιατί δεν επιθυμούμε το ρομπότ να κάνει πολλές στροφές αλλά να ακολουθεί μία πιο ομαλή-ευθεία πορεία.

\[ w_{turn} = \sum_i^{PathSize} θi \]

Τελευταίο κόστος το οποίο πρέπει να υπολογιστεί είναι το κόστος κάλυψης το οποίο θα ισούται με:

\[ w_{cove} = 1 - \frac{\sum_{i=1}^{PathSize} Coverage[x_{p_{i}}, y_{p_{i}}]}{PathSize*255} \]

Αφού υπολογιστούν όλα τα παραπάνω κόστη, ακολουθεί κανονικοποίηση: 

\[ w_{dist_{n}}^k = 1 - \frac{w_{dist}^k-min(w_{dist})}{max(w_{dist})-min(w_{dist})} \]
\[ w_{turn_{n}}^k = 1 - \frac{w_{turn}^k-min(w_{turn})}{max(w_{turn})-min(w_{turn})} \]
\[ w_{cove_{n}}^k = 1 - \frac{w_{cove}^k-min(w_{cove})}{max(w_{cove})-min(w_{cove})} \]
\[ w_{topo_{n}}^k = 1 - \frac{w_{topo}^k-min(w_{topo})}{max(w_{topo})-min(w_{topo})} \]

Τελικά, χρησιμοποιείται η ακόλουθη έκφραση για το τελικό κόστος:

\[ W = 2^3*w_{topo} + 2^2*w_{dist} + 2*w_{cove} + w_{turn} \]

Προκειμένου να γίνουν όλα τα παραπάνω προστέθηκε στο αρχείο \textlatin{autonomous\_expl.yaml} η παρακάτω γραμμή κώδικα:

\textlatin{
    \lstinputlisting[language=Python,firstline=140,lastline=140, basicstyle=\tiny]{./code.txt}
}

Στη συνέχεια στο αρχείο \textlatin{target\_selection.py} δημιουργήθηκε η μέθοδος \textlatin{selectSmartTarget} η οποία υλοποιεί όλη τη διαδικασία που περιγράφηκε παραπάνω.

\textlatin{
    \lstinputlisting[language=Python,firstline=143,lastline=210, basicstyle=\tiny]{./code.txt}
}

Τώρα, προκειμένου να κληθεί η παραπάνω συνάρτηση έπρεπε να προστεθεί και ένα μικρό κομμάτι στη συνάρτηση \textlatin{selectTarget} όπως φαίνεται παρακάτω:

\textlatin{
    \lstinputlisting[language=Python,firstline=213,lastline=232, basicstyle=\tiny]{./code.txt}
}

%--- Extra Challenge 1 ---
\subsection{\textlatin{Extra Challenge 1: Path optimization / alteration.}}

Προκειμένου να γίνει η διαδρομή πιο ομαλή επιλέχθηκε να εφαρμοσθεί η τεχνική \textlatin{Minimization via Gradient descent} όπως αυτή περιγράφεται στην παρουσίαση 8. Δηλαδή, αν $X$ το αρχικό μονοπάτι, $x_i$ ένα σημείο του πρώτου μονοπατιού και $Y$ μία δεύτερη καμπύλη, $y_i$ ένα σημείο του δεύτερου μονοπατιού αντίστοιχα, τότε προκειμένου να ελαχιστοποιήσω τις:

\[ f = (x_i - y_i)^2 \]
\[ g = ( y_i-y_{i+1})^2 \]

χρησιμοποιώ τον \textlatin{Gradient Descent}:

\[ GD: y_i = y_i + a *(x_i - y_i) + b* (y_{i+1} - 2y_i + y_{i-1}), \text{ με } 1 \leq i \leq N-1 \]

έως ότου το παρακάτω άθροισμα να συγκλίνει:

\[ \sum_{i=1}^{N-1} a *(x_i - y_i) + b* (y_{i+1} - 2y_i + y_{i-1})  < 10^{-3} \]

Η παραπάνω διαδικασία πραγματοποιήθηκε με την προσθήκη μέρος κώδικα στη συνάρτηση \textlatin{selectTarget}:


\textlatin{
    \lstinputlisting[language=Python,firstline=237,lastline=255, basicstyle=\tiny]{./code.txt}
}